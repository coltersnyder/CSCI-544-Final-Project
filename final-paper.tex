
\documentclass[conference]{IEEEtran}
\usepackage[
backend=biber,
style=ieee,
]{biblatex}
\addbibresource{citations.bib} %Imports bibliography file
\IEEEoverridecommandlockouts
% The preceding line is only needed to identify funding in the first footnote. If that is unneeded, please comment it out.
%\usepackage{cite}
\usepackage{amsmath,amssymb,amsfonts}
\usepackage{algorithmic}
\usepackage{graphicx}
\usepackage{textcomp}
\usepackage{xcolor}
\def\BibTeX{{\rm B\kern-.05em{\sc i\kern-.025em b}\kern-.08em
    T\kern-.1667em\lower.7ex\hbox{E}\kern-.125emX}}
\begin{document}

\title{Study of the Implementation of Basic Snow Print Depression in OpenGL}

\author{\IEEEauthorblockN{Elijah Mt. Castle}
\IEEEauthorblockA{\textit{Department of Computer Science} \\
\textit{Colorado School of Mines}\\
Golden, Colorado, United States \\
mtcastle@mines.edu}
\and
\IEEEauthorblockN{Colter Snyder}
\IEEEauthorblockA{\textit{Department of Computer Science} \\
\textit{Colorado School of Mines}\\
Golden, Colorado, United States \\
csnyder1@mines.edu}
}

\maketitle

\begin{abstract}
Realistic snow physics has been a dream in computer graphics since we first could make 3D graphics. This is now possible to a large degree all thanks to tessellation shaders. It was not long after tessellation shaders were introduced that accurate snow physics were being created. This paper seeks to explore the most basic type of snow simulation through the study of snow print depression.
\end{abstract}

\begin{IEEEkeywords}
graphics, OpenGL, snow, simulation, tessellation
\end{IEEEkeywords}

\section{Introduction}
Since the early days of computer graphics, snow physics took shape in the form of footprint textures that would show when the character would walk but beyond that no depressions could be seen. Then, tessellation shaders were introduced in 2010 and this changed the field forever. In 2013, the first demo of mud depressions were made, this eventually transformed into a full game that implemented mud physics, then a sequal implementing snow physics. Many other studios have made amazing snow physics including Pixar with Frozen and Rockstar Games with Red Dead Redemption 2.

\section{Definition of Terms}


\section{Background}


\section{Implementation}


\section{Limitations}


\section{Future Work}


\section{Conclusion}


\printbibliography

\end{document}
